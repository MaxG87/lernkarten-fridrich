\documentclass[12pt,a4paper,landscape]{scrartcl}
% evtl. nützliche Pakete
\usepackage{amsmath}
\usepackage{amssymb}
\usepackage{booktabs}
\usepackage[ngerman]{babel}
\usepackage[right]{eurosym} % mit `\euro' oder `\euro{Betrag}' zu benutzen
\usepackage[T1]{fontenc}
\usepackage{fontspec}
\usepackage[top=3.0cm,right=3.0cm,bottom=3.0cm,left=3.0cm]{geometry}
\usepackage{graphicx}
\usepackage{hyperref} % verlinkt automatisch alle Referenzen, sodass am Rechner
                      % mit Mausklicken dort hin gesprungen wird.
\usepackage[utf8]{inputenc}
\usepackage{lmodern}
\usepackage{mdwlist} % ermöglicht mit z.B. itemize* den Zeilenabstand in
                     % Aufzählungen kleiner zu machen
\usepackage{pdfpages}

\renewcommand{\arraystretch}{1.5}

%Abkürzungen langer Befehle
\renewcommand \( {\left (}
\renewcommand \) {\right )}
\renewcommand \[ {\left [}
\renewcommand \] {\right ]}
\newcommand{\gdw}[0]{\ensuremath{\Longleftrightarrow}}
\newcommand{\E}[0]{\ensuremath{\mathbb{E}}}
\newcommand{\N}[0]{\ensuremath{\mathbb{N}}}
\newcommand{\Q}[0]{\ensuremath{\mathbb{Q}}}
\newcommand{\R}[0]{\ensuremath{\mathbb{R}}}
% \DeclareMathOperator{\Acc}{Acc}

%Sonstiges
% \addtolength{\headheight}{1\baselineskip} %vergrößert eingeplante Kopfhöhe.
% Bei vielen Zeilen (4 ist viel) wird dadurch verhindert, das der Kopf in den
% Text rutscht
\title{HIER TITEL EINFÜGEN}
\author{Max Görner}
\date{\today}

\newlength{\cellwidth}
\setlength{\cellwidth}{4.5cm}
\newlength{\cellheight}
\setlength{\cellheight}{1cm}
\newcommand{\pllimg}[1] {\includegraphics{PLL#1.png}}
\newcommand{\pllalgo}[1] {
    \begin{minipage}{\cellwidth}
        \tiny
        \texttt{
            \input{pll#1.tex}
        }
    \end{minipage}
}

\begin{document}

\begin{tabular}{p{\cellwidth}|p{\cellwidth}|p{\cellwidth}|p{\cellwidth}|p{\cellwidth}}
    \pllimg{01} & \pllimg{02} & \pllimg{03} & \pllimg{04} & \pllimg{05} \\\hline
    \pllimg{06} & \pllimg{07} & \pllimg{08} & \pllimg{09} & \pllimg{10} \\\hline
    \pllimg{11} & \pllimg{12} & \pllimg{13} & \pllimg{14} & \pllimg{15} \\\hline
    \pllimg{16} & \pllimg{17} & \pllimg{18} & \pllimg{19} & \pllimg{20} \\\hline
    \pllimg{21} \\\hline
\end{tabular}

\newpage

\begin{tabular}{p{\cellwidth}|p{\cellwidth}|p{\cellwidth}|p{\cellwidth}|p{\cellwidth}}
    \pllalgo{01} & \pllalgo{02} & \pllalgo{03} & \pllalgo{04} & \pllalgo{05} \\\hline
    \pllalgo{06} & \pllalgo{07} & \pllalgo{08} & \pllalgo{09} & \pllalgo{10} \\\hline
    \pllalgo{11} & \pllalgo{12} & \pllalgo{13} & \pllalgo{14} & \pllalgo{15} \\\hline
    \pllalgo{16} & \pllalgo{17} & \pllalgo{18} & \pllalgo{19} & \pllalgo{20} \\\hline
    \pllalgo{21} \\\hline
\end{tabular}


\end{document}
